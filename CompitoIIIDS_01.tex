
%\documentclass[12pt]{report}
%\documentclass[12pt]{extreport}
\documentclass[14pt]{extarticle}
%\documentclass{memoir}

\usepackage{graphicx}
\usepackage{setspace}
\usepackage{amsmath,amssymb}
\usepackage{IEEEtrantools}
\usepackage{cancel}
\usepackage[font=small,labelfont=bf]{caption}
\usepackage{enumitem}

\usepackage{verbatim}
%\usepackage{textcomp}
\usepackage{eurosym}


\usepackage[T1]{fontenc}
\usepackage[utf8]{inputenc}
\usepackage[italian]{babel}


%\usepackage{imakeidx}%
%\makeindex[program=xindy]%, options=-C utf8 -L portuguese]%
\usepackage{makeidx}
\makeindex




\usepackage{geometry}
 \geometry{
 a4paper,
 total={170mm,264mm},
 left=8mm,
 right=8mm,
 top=10mm,
 bottom=10mm
 }

\begin{document}

%\backmatter
%text\index{test}
%\printindex

\pagenumbering{gobble}

\begin{center}
{\bf Compito in classe di fisica}
\end{center}

\begin{enumerate}
	\item Quale (o quali!) delle seguenti espressioni \emph{può} essere corretta?
	
		$(a)\quad\omega = \sqrt{\frac{k}{m}}
		\qquad (b)\quad T = \sqrt{\frac{k}{m}}
		\qquad (c)\quad \omega = \sqrt{\frac{F}{mvt}}$
			 
		Giustificare la risposta data facendo uso del calcolo dimensionale.
		
	\item Come occorre modificare le equazioni del moto uniformemente accelerato
	
	$$
		x = x_0 + v_0t +\frac{1}{2}at^2\qquad
		v = v_0 + at
	$$
	affinchè esse risultino le equazioni del moto circolare uniformemente accelerato?
	
	\item Un sasso, vincolato a una fune, viene fatto girare, e viene ripreso da una telecamera. Dopo $2ms$ il sasso ha compiuto un angolo di 40 gradi. Sapendo che la lunghezza della corda è $1.3$ $m$, calcolare la lunghezza dell'arco di traiettoria compiuto dal sasso.
	
	\item Un punto materiale vincolato a una molla ha un periodo di oscillazione di $23$ $ms$ e una ampiezza $5$ $cm$. Ipotizzando che al tempo $t = 0s$ passi per il punto di equilibrio, calcolare la posizione $x$, la velocità $v$ e l'accelerazione $a$ dopo un tempo pari a $1.1s$.

	\item Un punto materiale vincolato a una molla ha un periodo di oscillazione di $23$ $ms$ e una ampiezza $5$ $cm$. Ipotizzando che al tempo $t = 0s$ passi per il punto di equilibrio, calcolare il tempo $t$ necessario perchè il punto materiale arrivi al punto di coordinata $x = 2,5$ $cm$. Calcolare il tempo necessario perchè arrivi al punto di coordinata $x = -2,5$ $cm$.
	
	\item Quanto vale un angolo $\theta$ sapendo che $\cos(\theta) = 3.2$? E quanto vale $\theta$, nel caso $\tan(\theta) = 3.2$? Esiste un solo valore di $\theta$ tale che può soddisfare l'ultima delle due condizioni? Se ce ne sono altri, quali possono essere?

	\item Nel triangolo di figura \ref{fig:triangolo}, l'angolo in $A$ misura $\pi/6$ radianti, il lato $AB$ misura $4cm$ e il lato $BC$ misura $1.1cm$. Calcolare perimetro e area.
\end{enumerate}

\begin{figure}[h!]		
	\centering
   	\includegraphics[width=1.8in]{triangolo.png}
  	\caption{triangolo}
   	\label{fig:triangolo}
\end{figure}

\end{document}